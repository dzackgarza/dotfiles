\newcommand{\dgens}[1]{\gens{\gens{ #1 }}}
\newcommand{\ctz}[1]{\, {\converges{{#1} \to\infty}\longrightarrow 0} \, }
\newcommand{\conj}[1]{{\overline{{#1}}}}
\newcommand{\complex}[1]{{ {#1}_{\bullet}} }
\newcommand{\cocomplex}[1]{ { {#1}^{\bullet}} }

\newcommand{\floor}[1]{{\left\lfloor #1 \right\rfloor}}
\newcommand{\ceiling}[1]{{\left\lceil #1 \right\rceil}}
\newcommand{\fourier}[1]{\widehat{#1}}
\newcommand{\embedsvia}[1]{\xhookrightarrow{#1}}
\newcommand{\openimmerse}[0]{\underset{\scriptscriptstyle O}{\hookrightarrow}}
\newcommand{\weakeq}[0]{\underset{\scriptscriptstyle W}{\rightarrow}}

\newcommand{\fromvia}[1]{\xleftarrow{#1}}
\newcommand{\generators}[1]{\left\langle{#1}\right\rangle}
\newcommand{\gens}[1]{\left\langle{#1}\right\rangle}
\newcommand{\globsec}[1]{{{\Gamma}\qty{#1} }}
\newcommand{\Globsec}[1]{{{\Gamma}\qty{#1} }}
\newcommand{\langL}[1]{ {}^{L}{#1} }

\newcommand{\equalsbecause}[1]{\overset{#1}{=}}
\newcommand{\congbecause}[1]{\overset{#1}{\cong}}
\newcommand{\congas}[1]{\underset{#1}{\cong}}
\newcommand{\isoas}[1]{\underset{#1}{\cong}}
\newcommand{\addbase}[1]{{ {}_{\pt} }}
\newcommand{\ideal}[1]{\mathcal{#1}}
\newcommand{\adjoin}[1]{ { \left[ \scriptstyle {#1} \right] } }
\newcommand{\polynomialring}[1]{ { \left[ {#1} \right] } }
\newcommand{\htyclass}[1]{ { \left[ {#1} \right] } }
\newcommand{\qtext}[1]{{\quad \operatorname{#1} \quad}}
\newcommand{\abs}[1]{{\left\lvert {#1} \right\rvert}}
\newcommand{\stack}[1]{\mathclap{\substack{ #1 }}} 

\newcommand{\powerseries}[1]{ { \left[ {#1} \right] } }
\newcommand{\functionfield}[1]{ { \left( {#1} \right) } }
\newcommand{\rff}[1]{ \functionfield{#1} }

\newcommand{\fps}[1]{{\left[\left[ #1 \right]\right]  }}
\newcommand{\formalseries}[1]{ \fps{#1} }
\newcommand{\formalpowerseries}[1]{ \fps{#1} }
\newcommand\fls[1]{{\left(\left( #1 \right)\right)  }}

\newcommand\lshriek[0]{{}_{!}}
\newcommand\pushf[0]{{}^{*}}

\newcommand{\nilrad}[1]{{\sqrt{0_{#1}} }}
\newcommand{\jacobsonrad}[1]{{J ({#1}) }}
\newcommand{\localize}[1]{ \left[ { \scriptstyle { {#1}\inv}  } \right]}
\newcommand{\primelocalize}[1]{ \left[ { \scriptstyle { { ({#1}^c) }\inv}  } \right]}
\newcommand{\plocalize}[1]{\primelocalize{#1}}
\newcommand{\sheafify}[1]{ \left( #1 \right)^{\scriptscriptstyle \mathrm{sh}} }
\newcommand{\complete}[1]{{ {}_{ \hat{#1} } }}
\newcommand{\takecompletion}[1]{{ \overbrace{#1}^{\widehat{\hspace{4em}}}  }}
\newcommand{\pcomplete}[0]{{ {}^{ \wedge }_{p} }}
\newcommand{\kv}[0]{{ k_{\hat{v}} }}
\newcommand{\Lv}[0]{{ L_{\hat{v}} }}

\newcommand{\twistleft}[2]{{ {}^{#1} #2 }}
\newcommand{\twistright}[2]{{ #2 {}^{#1} }}
\newcommand{\liesover}[1]{{ {}_{/ {#1}} }}
\newcommand{\liesabove}[1]{{ {}_{/ {#1}} }}
\newcommand{\slice}[1]{_{/ {#1}} }
\newcommand{\coslice}[1]{_{{#1/}} }
\newcommand{\quotright}[2]{ {}^{#1}\mkern-2mu/\mkern-2mu_{#2} }
\newcommand{\quotleft}[2]{ {}_{#2}\mkern-.5mu\backslash\mkern-2mu^{#1} }
\newcommand{\invert}[1]{{ \left[ { \scriptstyle \frac{1}{#1} } \right] }}
\newcommand{\symb}[2]{{ \qty{ #1 \over #2 } }}
\newcommand{\squares}[1]{{ {#1}_{\scriptscriptstyle \square} }} 

\newcommand\cartpower[1]{{ {}^{ \scriptscriptstyle\times^{#1} }  }}
\newcommand\disjointpower[1]{{ {}^{ \scriptscriptstyle\coprod^{#1} }  }}
\newcommand\sumpower[1]{{ {}^{ \scriptscriptstyle\oplus^{#1} }  }}
\newcommand\prodpower[1]{{ {}^{ \scriptscriptstyle\times^{#1} }  }}
\newcommand\tensorpower[2]{{ {}^{ \scriptstyle\otimes_{#1}^{#2} }  }}
\newcommand\tensorpowerk[1]{{ {}^{ \scriptscriptstyle\otimes_{k}^{#1} }  }}
\newcommand\derivedtensorpower[3]{{ {}^{ \scriptstyle {}_{#1} {\otimes_{#2}^{#3}} }  }}
\newcommand\smashpower[1]{{ {}^{ \scriptscriptstyle\smashprod^{#1} }  }}
\newcommand\wedgepower[1]{{ {}^{ \scriptscriptstyle\smashprod^{#1} }  }}
\newcommand\fiberpower[2]{{ {}^{ \scriptscriptstyle\fiberprod{#1}^{#2} }  }}
\newcommand\powers[1]{{ {}^{\cdot #1} }}
\newcommand\skel[1]{{ {}^{ (#1) } }}
\newcommand\transp[1]{{ \, {}^{t}{ \left( #1 \right) } }}

\newcommand{\inner}[2]{{\left\langle {#1},~{#2} \right\rangle}}
\newcommand{\inp}[2]{{\left\langle {#1},~{#2} \right\rangle}}
\newcommand{\poisbrack}[2]{{\left\{ {#1},~{#2} \right\} }}
